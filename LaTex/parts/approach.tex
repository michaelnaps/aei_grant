\section{Aim 2}

\subsection{Rationale}

In aim 2, the design of the end effector, and the core tubing is addressed. In this portion of the project, the end effector will be designed to house the TCP such that it is not exposed, can be replaced when necessary, and does not limit its range of motion. To achieve this, the four strands of TCP will be lined in parallel to the center tube, which contains the ground wire, as well as the camera. It should be noted that this task is not dependent on the completion on Aim 1, as the TCP wires can be replicated with past test wire, or regular thick string.

The core tubing which is extruded from the housing module, and connects to the end effector via a clip system will likewise need to house the necessary wires to the TCP, the ground wire, and the camera wiring. This portion of the tubing will be relatively simple in configuration, with the primary limiting factor being the diameter of the tube.

\begin{figure}[ht]
	\centering
	\includegraphics[scale=0.15]{prototype_assembly}
	\caption{Prototype Assembly Developed in 2021-2022 Academic Year by OSU Capstone Team}
	\label{fig:prototype_assembly}
\end{figure}

\subsection{Approach}

The end effector will be constructed with the intent of a two-layer configuration. The center will be comprised of the ground wire and the camera, similar to the prototype made during the OSU capstone project with an insulating tube surrounding the components (layer 1). Next, the TCP will be connected in parallel around the center shaft and connected to the common ground wire. Each string will also be connected to an individual active wire which is used to control the current supplied to the TCP from the main controller. The TCP will then be surrounded by another layer of insulated tubing (layer 2). It is also possible that a third intermediate layer will be introduced to insulate the individual TCP string from one another, this will hopefully be avoidable though as it would unnecessarily increase the diameter of the assembled system. This portion of the device will only span one to two inches in length. Note that the controller need not be completed for testing, in initial scenarios, a simple user-defined analog signal can be used to test motion.

The core tubing will be considerably simpler to assemble and can be stylized to a single tube housing the necessary wires which will each be individually insulated. This is similar to every day USB pin-out wiring and other related devices. Because of its simplicity, the core tubing is not likely to break or have technical issues after being assembled. For this reason, the end effector and core tubing will be design separately and connected together via a locking mechanism. This way, if the end effector malfunctions, it can be replaced relatively easily. It may also be necessary to incorporate a center-line down both components such that they can be attached to one another in a consistent configuration.

\subsection{Risks and Alternatives}

The largest risk being addressed in this portion of development is the hard constraint for the diameter of the tubing. The most complex portion of the tubing, the end effector, must fit within the confines of commonly used intubating tubes. For development purposes, the constraint can be relaxed to the smallest adult-specific tube available (8[mm] diameter) instead of tubes used in children. That said, in the event that the TCP and other components do not fit within these constraints, alternatives to this configuration will have to be researched.

\section{Aim 3}

\subsection{Rationale}

The primary goal of Aim 3 will be to integrate the on-board micro-controller with the neural network developed in the Spring of 2022 by the computer science and engineering capstone team. Their system utilizes a trained neural network to identify and trace the anatomical features of the airway from video records. The outputs of the identifier are boxes which encapsulate the feature, as well as a confidence interval value which relays how likely it is that the feature was identified correctly.

The automation component of the device is wholly dependent on the system discussed here. It will most likely be implemented on board the device, but depending on its weight and specifications, it may be run on a separate unit, with a more complex communication system. Either way, the device will be capable of this feature with the addition of the camera to the end effector.

\subsection{Approach}

The neural network referenced here is largely sitting in a completed state. It is currently trained on a set of human mannequin example videos and performs to a high degree of accuracy. Once access is granted for the human intubation procedure videos housed on the Ohio State University Wexner Medical Center computer system, the network can be retrained for real human anatomy. The work done by the CSE department capstone team will be used to streamline this process, as they setup the neural network layers efficiently and were able to identify the anatomical features from human mannequin footage. The same network can thus be used for the real system.

Once the network is trained, a communication system between the on-board micro-controller and the NN will be developed. The current plan is to evaluate the run-time needs of the neural network and quantify whether it can be run from within the housing module, or if it should be run externally and communicate via a wired connection. It should be noted that the control system from Aim 1 does not need to be finished to test this communication system. As long as the outputs from the network match the inputs to the controller, they can be developed in parallel. This means that at the beginning of both processes, the communication point will be defined thoroughly to alleviate any difficulties down the road.

\subsection{Risks and Alternatives}

The largest risk in the development of the neural network is defining the level of confidence which the controller will allow before deciding a motion is safe. In other words, how low of a confidence is acceptable before the controller refuses to move in that direction. Setting this value too low factors in an obvious level of risk to the patient, as the device may navigate towards incorrectly identified features. Setting this setting too high may put the device at risk of stalling, i.e. reaching a point where no features are identifiable and not being able to make a decision. Ideally, a fail-safe will be implemented to account for this but it is also important to remember that the device as a whole is being designed with the intent that the practitioner is never absent from the procedure. They are required to watch the approach of the robot into the airway, and in these rare instances they would easily be able to manually override the neural network and move the robot in the proper direction.