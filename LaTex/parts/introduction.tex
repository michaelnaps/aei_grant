\section{Introduction}
\label{sect:introduction}

Endotracheal intubation is one of the most fundamental procedures within the hospital environment. It is the third most common procedure performed annually with an average between 13-20 million completed procedures \cite{hemmerling_first_2012}. With that said, the first pass success (FPS) rate of this procedure can drop as low as 70\% in the emergency setting, with each subsequent pass increasing the risk of damage to the patient 4-fold according to a 2014 study \cite{bernhard_first_2015}. These risks can be further increased by the presence of, but not limited to, a generally unhealthy airway, unexpected oxygen desaturation, larygospasm, and hypotension. The current process for intubation is characterized by the use of a direct, or video laryngoscope, or a flexible intubation scope. All of which prioritize increasing the visualization of the anatomical features of the mouth. To note, none of the current methods directly assist in guidance of the intubation tube during the procedure, with the exception of the flexible intubation scope.

The developmental research discussed here is being conducted in continuance of an Ohio State University capstone project which was completed in the 2021-2022 academic year. The goal of the project was to create an automated endotracheal intubation system which would have the capabilities of being self-driven via a trained neural network (NN). The novelty of the device is primarily in its actuation, which consist of four twisted and coiled polymer (TCP) wires. These wires behave similarly to tension driven actuators, commonly seen in continuum robots, with the exception that their power source is direct electrical current. This allows for the TCP to be actuated independent of location with reference to the housing module, unlike other continuum systems on the market. Furthermore, because the TCP is actuated via direct current, the housing module can be created to be much smaller than its tension driven counterparts.

The intended impact of this device is to increase the FPS rate by implementing the visualization techniques utilized in the flexibile intubation scope, along with an manual and automated guidance system. The system will be able to quickly alternate between the two modes, but will in either case prove to be wholistically beneificial to the patient's safety.