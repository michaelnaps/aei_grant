\section{Modeling Techniques}
\label{sect:modeling_techniques}
	
	As derived in \textit{Modeling of twisted and coiled polymers (TCP) muscle based on phenomenological approach} by Farzad Karami and Yonas Tadesse. For modeling of TCP actuators, the following steps are used to approximate the temperature and displacement of the material based on a given current, TCP dimensions and load.

	The change in resistance with respect to temperature is shown in Equation \ref{eq:tcp_resistance} below.
	
	\begin{equation}
	\label{eq:tcp_resistance}
		R(T) = R_{0} (1 + \alpha (T - T_{0}))
	\end{equation}
	
	Where $R(T)$ represents the current resistance in $[ohms]$, $R_{0}$ is the resistance at a reference temperature, $T$ is the current temperature of the TCP, and $T_{0}$ is the reference temperature. Finally, $\alpha$ is the coefficient of thermal expansion (CTE). For the purpose of manipulating tension using the TCP, a material with a negative CTE was chosen. It is further explored that the elastic modulus (E) is also determined by temperature.
	
	For equation \ref{eq:tcp_resistance} to be used accurately during modeling the change in temperature must also be approximated. This is completed below.
	
	\begin{equation}
	\label{eq:tcp_temperature}
		T - T_{\infty} =
			- \frac{R_{0} i^{2}}{-h A + R_{0} i_{2} \alpha}
			\left(
				1 -
				\exp^{\frac{-h A + R_{0} i_{2} \alpha}{m c_{p}}} t
			\right)
	\end{equation}
	
	Where $T_{\infty}$ is the ambient temperature, $i$ is the operating current, $h$ is the coefficient of heat convection, $A$ is the exposed TCP surface area, $m$ is the TCP mass, $c_{p}$ is the specific heat capacity, and $t$ is the time-passed. This is used to approximate the temperature, $T$, of the TCP with respect to a given current, $i$, and time, $t$.
	
	Using Equation \ref{eq:tcp_temperature} in combination with the function for $E$ (equation \ref{eq:tcp_elastic_modulus} and along with the dimensions of the TCP material, the displacement of the TCP can be calculated using the set of equations shown below.
	
	\begin{equation}
	\label{eq:tcp_elastic_modulus}
		E(T) = a_{1} T^{a_{2}} + a_{3}
	\end{equation}
	
	\begin{equation}
	\label{eq:tcp_displacement_elastic}
		\Delta_{el} = \frac{F}{k}
	\end{equation}
	
	\begin{equation}
	\label{eq:tcp_displacement_thermal}
		\Delta_{th} = \delta H
		\begin{matrix}			
			= \frac{L \delta L - \pi D \delta D}{\sqrt{L^{2} - \pi D^{2}}} \\
			= \frac{L^{2}_{0}}{H_{0}} \alpha_{L} (T - T_{0}) - \frac{pi D^{2}_{0}}{H_{0}} \alpha_{T} (T - T_{0})
		\end{matrix}
	\end{equation}
	
	\begin{equation}
	\label{eq:tcp_displacement}
		\Delta H = H_{0} - \Delta_{th} + \Delta_{el}
	\end{equation}
	
	Equation \ref{eq:tcp_elastic_modulus} represents a fitted function for the elastic modulus of a single string of TCP at a given temperature. Equation \ref{eq:tcp_displacement_elastic} is the displacement of the TCP with respect to the load and equation \ref{eq:tcp_displacement_thermal} is the displacement based on the CTE. Finally, equation \ref{eq:tcp_displacement} is the total displacement based on the previous two equations.
	
	Here, $a_{1}$, $a_{2}$ and $a_{3}$ are the fitted curve coefficients, $F$ is the load on the TCP, $k$ is the elastic coefficient (not derived here), $L_{0}$ is the initial TCP stretched length, $H_{0}$ is the initial TCP coiled length, $D_{0}$ is the diameter of the coiled wire, $\alpha_{L}$ is the longitudinal CTE, and $\alpha_{T}$ is the transverse CTE. The suse of these equations allow for the approximation of the TCP length at a given point in time.
	
	The issue that arises here is the assumption that the load, $F$, is constant. For the purpose of implementation in AEI, the tension must be dynamic in the modeling. This issue will be addressed during the modeling 